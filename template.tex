%%%%%%%%%%%%%%%%%%%%%%%%%%%%%%%%%%%%%%%%%
% Twenty Seconds Resume/CV
% LaTeX Template
% Version 1.0 (14/7/16)
%
% Original author:
% Carmine Spagnuolo (cspagnuolo@unisa.it) with major modifications by 
% Vel (vel@LaTeXTemplates.com) and Harsh (harsh.gadgil@gmail.com)
%
% License:
% The MIT License (see included LICENSE file)
%
%%%%%%%%%%%%%%%%%%%%%%%%%%%%%%%%%%%%%%%%%

%----------------------------------------------------------------------------------------
%	PACKAGES AND OTHER DOCUMENT CONFIGURATIONS
%----------------------------------------------------------------------------------------

\documentclass[letterpaper]{twentysecondcv} % a4paper for A4

% Command for printing skill overview bubbles
\newcommand\skills{ 
~
	\smartdiagram[bubble diagram]{
        \textbf{Data}\\\textbf{~Science~},
        Statistical\\Analysis,
        Predictive\\modelling,
        Data\\Visualization,
        Geospatial\\Analysis,
        Machine\\Learning
    }
}

% Programming skill bars
\programming{{\faIcon[regular]{globe} - Geospatial analysis tools / 6}, {\faIcon[regular]{r-project} - R \large$\bullet$ \normalsize \faGithub - Git \large$\bullet$ \normalsize Agile framework / 5.4}, {\faIcon[regular]{database} - SQL \large$\bullet$ \large \LaTeX / 3.7}, {\faPython \hspace{1mm}- Python / 1.7}}
% Languages bars
\Languages{{Portuguese / 6}, {English / 5.2}, {German / 2.2}}

% Projects text
\education{
\textbf{PhD. in Phys. Geography (2016 - 22)} \\
%Topic: Statistical and ML methods for natural hazards spatial prediction\\
University of Vienna - UNIVIE \\

\textbf{MsC. in Phys. Geography (2013 - 15)} \\
%Topic: Statistical and ML methods for natural hazards spatial prediction\\
UFRJ, Rio de Janeiro, BR \\

%\textbf{MSc., Geography} \\
%Specialization: Physical Geography and Urban Planning \\
%Federal Univ. of Rio de Janeiro - \textbf{UFRJ} \\
%2013 - 2015 | Rio de Janeiro, Brazil

%\textbf{BSc., Biology} \\
%Federal Fluminense Univ. - \textbf{UFF}\\
%2007 - 2011 | Rio de Janeiro, Brazil
}

%----------------------------------------------------------------------------------------
%	 PERSONAL INFORMATION
%----------------------------------------------------------------------------------------
% If you don't need one or more of the below, just remove the content leaving the command, e.g. \cvnumberphone{}

\cvname{Pedro Henrique Muniz Lima} % Your name
\cvjobtitle{ Data Analyst/Scientist } % Job
% title/career

\cvlinkedin{/in/pedrolima-ds/}
%\cvgithub{munizlimap15}
\cvnumberphone{(43) 680 1152366}    % Phone number
\cvsite{My website - \footnotesize \textit{\textcolor{gray}{under construction}}}             % Personal website /munizlimap15.github.io/Pedrolima
\normalsize
\cvmail{pedrohe@gmail.com}          % Email address

%----------------------------------------------------------------------------------------

\begin{document}

\makeprofile % Print the sidebar
 
%----------------------------------------------------------------------------------------
%	 EXPERIENCE
%----------------------------------------------------------------------------------------

\section{Professional experiences}

\begin{twenty} % Environment for a list with descriptions
\twentyitem
    	{Sep 2021 -}
		{now}
        {Position: Researcher \color{black}\faIcon[duotone]{university}} 
        {\href{https://geographie.univie.ac.at/arbeitsgruppen/engage-geomorphologische-systeme-und-risikoforschung/}{\normalsize  \textbf{\underline{University of Vienna}}}}
        {}
        {\begin{itemize}
        \item Currently working as a researcher within the MoNOE project (\textit{Methodenentwicklung für die Gefährdungsmodellierung von Massenbewegungen in Niederösterreich}) at the University of Vienna.
        %\item It is a research project in cooperation wit the Geological survey and the Urban planning department of Lower Austria which aims to re-evaluate a natural hazards maps created to the Lower Austrian province a decade ago.
        %\item Varied Data Science related task, like prediction of upcoming landslides using statistical and machine learning methods, varied statistical and quantitative analyses of a large database of natural hazards occurrence, and etc...
        \item Re-evaluation of old landslide prediction model currently used in spatial planning and urban development, over newer landslide data to determine the quality of old predictive models.
        \item Integration of large database of landslide in a newer landslide predictive model using statistical predictive modelling.Including data handling, modelling, validation and interpretation. 
        \item Publication writing and conference participation. 
        \item \textbf{Main tech tools used}: ArcGIS, R, QGIS and Git. 
        \end{itemize}}
        \vspace{7mm}\\
        


\twentyitem
    	{Sep 2019 -}
		{Sep 2021}
        {Position: Data Scientist \faIcon[regular]{globe}} 
        {\href{http://www.ubiq.ai/}{\normalsize \textbf{\underline{Ubiq}}}}
        {}
        {\begin{itemize}
        %\item In Ubiq I was the DS responsible to the elaboration of spatial and temporal dynamic models for shared mobility demand-prediction. This models were integrated into an APP which displayed to the final users a real time demand-prediction of shared mobility automobiles (e.g., cars, scooters, mopeds)
        \item Elaboration of spatial and temporal dynamic models for shared mobility demand-prediction. 
        \item Experiences on building, maintaining and optimizing predictive models for car and moped fleets in cities like Berlin, Budapest, Viena, Dubai, Washington DC, between others. \item Large database pre-processing, engineering and preparation to building demand-prediction algorithms.
        \item Responsible to deliver a ready-to-go model to the back end team for deployment.
        \item Elaboration of KPIs and other impact measurement assessments.
        \item Historic data analysis for reports and presentations with clients.
        \item Storytelling and data visualization.
        \item Large datasets handling, management and information collection. 
        \item Participation in hiring processes. 
        \item \textbf{Main tech tools used}: R, SQL, FME, QGIS and Git.
        \end{itemize}}
        \\
	
    %\twentyitem
   		%{Sep 2015 -}
		%{Dec 2016}
        %{\textcolor{red}{Graduate Teaching Assistant}}
        %{\href{http://www.uoguelph.ca}{University of Guelph}}
        %{}
        %{
        %{\begin{itemize}
        %\item \textcolor{red}{Item 1}
        %\item \textcolor{red}{Item 2}
    %\end{itemize}}
        %}
     \\
    %\twentyitem
   		%{Sep 2015 -}
		%{Dec 2016}
        %{\textcolor{red}{Graduate Teaching Assistant}}
        %{\href{http://www.uoguelph.ca}{University of Guelph}}
        %{}
        %{
        %{\begin{itemize}
        %\item \textcolor{red}{Item 1}
        %\item \textcolor{red}{Item 2}
    %\end{itemize}}
        %}
        
	%\twentyitem{<dates>}{<title>}{<location>}{<description>}
\end{twenty}

%----------------------------------------------------------------------------------------
%	 RESEARCH
%----------------------------------------------------------------------------------------
\section{Research}
\begin{twenty}
	\twentyitem
    	{2016 - 2021}
		{}
        {PhD \faIcon[duotone]{university}}
        {\href{https://geographie.univie.ac.at/arbeitsgruppen/engage-geomorphologische-systeme-und-risikoforschung/}{\normalsize  \textbf{\underline{ENGAGE group - UNIVIE}}}}
        {}
        {
       	\hspace{1mm}
       	
       	\textbf{Thesis}: \textit{Landslide prediction mapping at varied scales.Methodological designs adaptations to better cope with common input data-related challenges.}
        {\begin{itemize}
        \item Used multiple statistical and machine learning algorithms to spatially predict natural hazard phenomena (landslides) in varied study sites. 
        \item Design methodological frameworks aiming to better cope with insufficient input datasets (e.g., bias, positional inaccuracy). 
        \item Participation in varied number of scientific conferences, including presentation and debate of modelling outcomes.
        \item Production of varied research items.
        \item \textbf{Tech tools}: R, ArcGIS, QGIS, \large \LaTeX
		\end{itemize}}
        }
\end{twenty}

\section{Other interests}
\begin{center} Summer $\textbullet$ Surf $\textbullet$ Outdoor activities $\textbullet$ Snowboard $\textbullet$ Maps $\textbullet$ Houseplants $\textbullet$ coffee $\textbullet$ Bike $\textbullet$ Watersports\\ 
\vspace{2mm}
%http://mirror.neu.edu.cn/CTAN/fonts/fontawesome/doc/fontawesome.pdf
\faCoffee \hspace*{0.5cm} \faBeer \hspace*{0.5cm}\faSpotify \hspace*{0.5cm}\faSunO \hspace*{0.5cm}\faBicycle \hspace*{0.5cm}\faGraduationCap  \hspace*{0.5cm} \faMapMarker \hspace*{0.5cm}\faMap \hspace*{0.5cm}\faPlane \hspace*{0.5cm}\faRoad \hspace*{0.5cm}\faSignal \hspace*{0.5cm} \faLineChart  \hspace*{0.5cm}\faUniversity 
\end{center}

 %\vspace{2mm}

\end{document} 
