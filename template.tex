%%%%%%%%%%%%%%%%%%%%%%%%%%%%%%%%%%%%%%%%%
% Twenty Seconds Resume/CV
% LaTeX Template
% Version 1.0 (14/7/16)
%
% Original author:
% Carmine Spagnuolo (cspagnuolo@unisa.it) with major modifications by 
% Vel (vel@LaTeXTemplates.com) and Harsh (harsh.gadgil@gmail.com)
%
% License:
% The MIT License (see included LICENSE file)
%
%%%%%%%%%%%%%%%%%%%%%%%%%%%%%%%%%%%%%%%%%

%----------------------------------------------------------------------------------------
%	PACKAGES AND OTHER DOCUMENT CONFIGURATIONS
%----------------------------------------------------------------------------------------

\documentclass[letterpaper]{twentysecondcv} % a4paper for A4

% Command for printing skill overview bubbles
\newcommand\skills{ 
~
	\smartdiagram[bubble diagram]{
        \textbf{Data}\\\textbf{~Science~},
        Statistical\\Analysis,
        Predictive\\modelling,
        Data\\Visualization,
        Geospatial\\Analysis,
        Machine\\Learning,
        Geomarketing
    }
}

% Programming skill bars
\programming{{\faIcon[regular]{globe} - Geospatial tools (ArcGIS Pro; Qgis)/ 6}, {\faIcon[regular]{r-project} - R \large$\bullet$ \normalsize \faGithub - Git \large$\bullet$ \normalsize Agile framework / 5.4}, {\faIcon[regular]{database} - SQL \large$\bullet$ \large \LaTeX / 3.7}, {\faPython \hspace{1mm}- Python %\large$\bullet$ \normalsize Power BI
/ 1.7}, {\normalsize Power BI - \footnotesize \textit{\textcolor{gray}{training started in 2024}}
/ 1.0}}
% Languages bars
\Languages{{Portuguese (Native)/ 6}, {English (C1)/ 5.2}, {German (B1) / 2.2}}

% Projects text
%\education{
%\textbf{PhD. in Phys. Geography (2016 - 22)} \\
%Topic: Statistical and ML methods for natural hazards spatial prediction\\
%University of Vienna - UNIVIE \\

%\textbf{MsC. in Phys. Geography (2013 - 15)} \\
%Topic: Statistical and ML methods for natural hazards spatial prediction\\
%UFRJ, Rio de Janeiro, BR \\

%\textbf{MSc., Geography} \\
%Specialization: Physical Geography and Urban Planning \\
%Federal Univ. of Rio de Janeiro - \textbf{UFRJ} \\
%2013 - 2015 | Rio de Janeiro, Brazil

%\textbf{BSc., Biology} \\
%Federal Fluminense Univ. - \textbf{UFF}\\
%2007 - 2011 | Rio de Janeiro, Brazil
%}

%----------------------------------------------------------------------------------------
%	 PERSONAL INFORMATION
%----------------------------------------------------------------------------------------
% If you don't need one or more of the below, just remove the content leaving the command, e.g. \cvnumberphone{}
\cvname{Pedro Henrique Muniz Lima, {\normalsize \textcolor{gray}{PhD}}} % Your name
\cvjobtitle{Data Analyst / Scientist} % Job
% title/career
\\
\cvlocation{\textbf{Location:} Vienna, Austria}
\\
\cvVISA{\textbf{VISA:} Open labor market access (EU).}
\\
\cvnation{\textbf{ Nationality:} Brazilian.}
\\


\cvlinkedin{/pedrolima-ds/}
\cvgithub{munizlimap15}
\cvnumberphone{(43) 680 1152366}    % Phone number
\cvsite{My website - \scriptsize \textit{\textcolor{gray}{under construction}}}             % Personal website /munizlimap15.github.io/Pedrolima
\normalsize
\cvmail{pedrohe@gmail.com}          % Email address

%----------------------------------------------------------------------------------------

\begin{document}

\makeprofile % Print the sidebar
 
%----------------------------------------------------------------------------------------
%	 EXPERIENCE
%----------------------------------------------------------------------------------------

\section{Relevant Professional experiences}

\begin{twenty} % Environment for a list with descriptions
\twentyitem
    	{Sep 2021 -}
		{Dec 2023}
        {Researcher; Principal Investigator (PI) \color{black}\faIcon[duotone]{university}} 
        {\href{https://geographie.univie.ac.at/arbeitsgruppen/engage-geomorphologische-systeme-und-risikoforschung/}{\normalsize  \textbf{\underline{University of Vienna}}}}
        {}
        {\begin{itemize}
        
        \item Researcher within the MoNOE project (\textit{Methodenentwicklung für die Gefährdungsmodellierung von Massenbewegungen in Niederösterreich}) at the University of Vienna (Department of Geography and Regional Research), in cooperation with the \textit{Land Niederösterreich}.
        \item Main activities include: Handling and integrating a landslide database into predictive modeling using statistical techniques. 
        \item Re-evaluation of old landslide prediction model currently used in spatial planning and urban development, over newer landslide data to determine the quality of old predictive models.\\
                 
        
        %\item \textbf{Main tech tools used}: 
        \mybox[fill=gray!20]{\normalsize R} \hspace{2mm}
        \mybox[fill=gray!20]{\normalsize ArcGIS Pro} \hspace{2mm}
        \mybox[fill=gray!20]{\normalsize QGIS}  \hspace{2mm} \mybox[fill=gray!20]{\normalsize Git} \hspace{2mm} 
        \end{itemize}}\\
        


\twentyitem
    	{Sep 2019 -}
		{Sep 2021}
        {Data Scientist \faIcon[regular]{globe}; GIS} 
        {\href{http://www.ubiq.ai/}{\normalsize \textbf{\underline{Ubiq}}}}
        {}
        {\begin{itemize}
        
        %\item In Ubiq I was the DS responsible to the elaboration of spatial and temporal dynamic models for shared mobility demand-prediction. This models were integrated into an APP which displayed to the final users a real time demand-prediction of shared mobility automobiles (e.g., cars, scooters, mopeds)
        \item Elaboration of spatial and temporal dynamic models for shared mobility demand-prediction. 
        \item Maintaining and optimizing data pipeline models for car and moped fleets in cities like Berlin, Budapest, Viena, Dubai, Washington DC, between others. \item Large database pre-processing, engineering and preparation to building demand-prediction algorithms.
        %\item Responsible to deliver a ready-to-go model to the back end team for deployment.
        \item Elaboration of KPIs and other impact measurement assessments.
        \item Historic data analysis for reports and presentations with clients.
        \item Storytelling and data visualization.
        
        \item Participation in hiring processes. \\
        
        %\item \textbf{Main tech tools used}: R, SQL, FME, QGIS and Git.
        \mybox[fill=gray!20]{\normalsize R} \hspace{2mm}
        \mybox[fill=gray!20]{\normalsize SQL}  \hspace{2mm} 
        \mybox[fill=gray!20]{\normalsize FME}  \hspace{2mm} 
        \mybox[fill=gray!20]{\normalsize QGIS}  \hspace{2mm} \mybox[fill=gray!20]{\normalsize Git} \hspace{2mm} 
        
        \end{itemize}}
	
    %\twentyitem
   		%{Sep 2015 -}
		%{Dec 2016}
        %{\textcolor{red}{Graduate Teaching Assistant}}
        %{\href{http://www.uoguelph.ca}{University of Guelph}}
        %{}
        %{
        %{\begin{itemize}
        %\item \textcolor{red}{Item 1}
        %\item \textcolor{red}{Item 2}
    %\end{itemize}}
        %}
    %\twentyitem
   		%{Sep 2015 -}
		%{Dec 2016}
        %{\textcolor{red}{Graduate Teaching Assistant}}
        %{\href{http://www.uoguelph.ca}{University of Guelph}}
        %{}
        %{
        %{\begin{itemize}
        %\item \textcolor{red}{Item 1}
        %\item \textcolor{red}{Item 2}
    %\end{itemize}}
        %}
        
	%\twentyitem{<dates>}{<title>}{<location>}{<description>}
\end{twenty}

%----------------------------------------------------------------------------------------
%	 RESEARCH
%----------------------------------------------------------------------------------------
\section{Relevant education and trainning}
\begin{twenty}
	\twentyitem
    	{2015 - 2022}
		{}
        {  PhD in Physical Geography\faIcon[duotone]{university}}
        {\href{https://geographie.univie.ac.at/arbeitsgruppen/engage-geomorphologische-systeme-und-risikoforschung/}{\normalsize  \textbf{\underline{ENGAGE group - UNIVIE}}}}
        {}
        {
       	\hspace{1mm}
       	
       	\textbf{Thesis}: \textit{"Landslide prediction mapping at varied scales.Methodological designs adaptations to better cope with common input data-related challenges".}
        {\begin{itemize}
        \small
        \item Used multiple statistical and machine learning algorithms to spatially predict natural hazard phenomena (landslides) in varied study sites. 
        \item Design methodological frameworks aiming to better cope with insufficient input datasets (e.g., bias, positional inaccuracy). 
        \item Participation in varied number of scientific conferences, including presentation and debate of modelling outcomes.
        \item Production of varied research items.
        \\
        
        %\item \textbf{Main tech tools used}: R, SQL, FME, QGIS and Git.
        \mybox[fill=gray!20]{\normalsize R} \hspace{2mm}
        \mybox[fill=gray!20]{\normalsize ArcGIS Pro}  \hspace{2mm} 
        \mybox[fill=gray!20]{\normalsize QGIS}  \hspace{2mm} 
        \mybox[fill=gray!20]{\normalsize \large \LaTeX}  \hspace{2mm} \mybox[fill=gray!20]{\normalsize Git} \hspace{2mm} \vspace{4mm}
        %\item \textbf{Main tech tools used}: R, ArcGIS Pro, QGIS, \large \LaTeX
		\end{itemize}}
        }
        \twentyitem
    	{2013 - 2015}
		{}
        {  MsC in Physical Geography \faIcon[duotone]{university}}
        {\href{https://ufrj.br/en/}{\normalsize  \textbf{\underline{UFRJ}}}}
        {}
        {
       
        {\begin{itemize}
        \small
        
        %\item \textbf{Main tech tools used}: R, SQL, FME, QGIS and Git.
        \mybox[fill=gray!20]{\normalsize ArcGIS}  \hspace{2mm} 
        \mybox[fill=gray!20]{\normalsize QGIS} \hspace{2mm} 
        \mybox[fill=gray!20]{\normalsize EXCEL} 
		\end{itemize}}          }
\end{twenty}

%\section{Varied certifications\faIcon[duotone]{university}} 
               
               
          \vspace{2mm}     
        \small
       \mybox[fill=gray!20]{\normalsize Linux and First Steps on the VSC-3 Cluster} 
        \hspace{2mm}  
        \mybox[fill=gray!20]{\normalsize Microsoft Power BI Data Analyst - Training} 
        \vspace{2mm}  
        \mybox[fill=gray!20]{\normalsize R - Advanced}
        \hspace{2mm}
        \mybox[fill=gray!20]{\normalsize International Summer School on Geospatial Data Science Using R} \hspace{2mm} 
        \mybox[fill=gray!20]{\small etc...}

\\

\section{Other interests}
\begin{center} Summer $\textbullet$ Surf $\textbullet$ Landscapes $\textbullet$ Outdoor activities $\textbullet$ Snowboard $\textbullet$ Travel $\textbullet$ Maps $\textbullet$ Houseplants $\textbullet$ Coffee $\textbullet$ Bike $\textbullet$ Watersports\\ 
\vspace{2mm}
%http://mirror.neu.edu.cn/CTAN/fonts/fontawesome/doc/fontawesome.pdf
\faCoffee \hspace*{0.5cm} \faBeer \hspace*{0.5cm}\faSpotify \hspace*{0.5cm}\faSunO \hspace*{0.5cm}\faBicycle \hspace*{0.5cm}\faGraduationCap  \hspace*{0.5cm} \faMapMarker \hspace*{0.5cm}\faMap \hspace*{0.5cm}\faPlane \hspace*{0.5cm}\faRoad \hspace*{0.5cm}\faSignal \hspace*{0.5cm} \faLineChart  \hspace*{0.5cm}\faUniversity 
\end{center}

 %\vspace{2mm}

\end{document} 
