%%%%%%%%%%%%%%%%%%%%%%%%%%%%%%%%%%%%%%%%%
% Twenty Seconds Resume/CV
% LaTeX Template
% Version 1.0 (14/7/16)
%
% Original author:
% Carmine Spagnuolo (cspagnuolo@unisa.it) with major modifications by 
% Vel (vel@LaTeXTemplates.com) and Harsh (harsh.gadgil@gmail.com)
%
% License:
% The MIT License (see included LICENSE file)
%
%%%%%%%%%%%%%%%%%%%%%%%%%%%%%%%%%%%%%%%%%

%----------------------------------------------------------------------------------------
%	PACKAGES AND OTHER DOCUMENT CONFIGURATIONS
%----------------------------------------------------------------------------------------

\documentclass[letterpaper]{twentysecondcv} % a4paper for A4

% Command for printing skill overview bubbles
\newcommand\skills{ 
~
	\smartdiagram[bubble diagram]{
        \textbf{Data}\\\textbf{~Science~},
        Statistical\\Analysis,
        Predictive\\modelling,
        Data\\Visualization,
        Geospatial\\Analysis,
        Machine\\Learning
    }
}

% Programming skill bars
\programming{{\faIcon[regular]{globe} - Geospatial analysis tools / 6}, {\faIcon[regular]{r-project} - R \large$\bullet$ \normalsize \faGithub - Git \large$\bullet$ \normalsize Agile framework / 5.4}, {\faIcon[regular]{database} - SQL \large$\bullet$ \large \LaTeX / 3.7}, {\faPython \hspace{1mm}- Python / 1.7}}
% Languages bars
\Languages{{Portuguese / 6}, {English / 5.2}, {German / 2.2}}

% Projects text
\education{
\textbf{PhD., Geography} \\
Topic: Statistical and ML methods for natural hazards spatial prediction\\
University of Vienna  - \textbf{UNIVIE} \\
2016 - 2022 | Vienna, Austria

%\textbf{MSc., Geography} \\
%Specialization: Physical Geography and Urban Planning \\
%Federal Univ. of Rio de Janeiro - \textbf{UFRJ} \\
%2013 - 2015 | Rio de Janeiro, Brazil

%\textbf{BSc., Biology} \\
%Federal Fluminense Univ. - \textbf{UFF}\\
%2007 - 2011 | Rio de Janeiro, Brazil
}

%----------------------------------------------------------------------------------------
%	 PERSONAL INFORMATION
%----------------------------------------------------------------------------------------
% If you don't need one or more of the below, just remove the content leaving the command, e.g. \cvnumberphone{}

\cvname{Pedro Henrique Muniz Lima} % Your name
\cvjobtitle{ Data Analyst/Scientist } % Job
% title/career

\cvlinkedin{/in/pedrolima-ds/}
%\cvgithub{munizlimap15}
\cvnumberphone{(43) 680 1152366}    % Phone number
\cvsite{github.io/Pedrolima/}             % Personal website /munizlimap15.github.io/Pedrolima
\cvmail{pedrohe@gmail.com}          % Email address

%----------------------------------------------------------------------------------------

\begin{document}

\makeprofile % Print the sidebar
 
%----------------------------------------------------------------------------------------
%	 EXPERIENCE
%----------------------------------------------------------------------------------------

\section{Professional experiences}

\begin{twenty} % Environment for a list with descriptions
\twentyitem
    	{Sep 2021 -}
		{now}
        {Position: Researcher \color{black}\faIcon[duotone]{university}} 
        {\href{https://geographie.univie.ac.at/arbeitsgruppen/engage-geomorphologische-systeme-und-risikoforschung/}{\normalsize  \textbf{\underline{\textcolor{mainblue}{University of Vienna}}}}}
        {}
        {\begin{itemize}
        \item Currently working as a researcher within the MoNOE project (\textit{Methodenentwicklung für die Gefährdungsmodellierung von Massenbewegungen in Niederösterreich}) at the University of Vienna.
        \item Re-evaluation of old landslide prediction model currently used in spatial planning and urban development, over newer landslide data to determine the quality of old predictive models.
        \item Integration of large database of landslide in a newer landslide predictive model using statistical predictive modelling.Including data handling, modelling, validation and interpretation. 
        \item Publication writing and conference participation. 
        \item \textbf{Main tech tools used}: ArcGIS, R, QGIS and Git. 
        \end{itemize}}
        \\

\twentyitem
    	{Sep 2019 -}
		{Sep 2021}
        {Position: Data Scientist \faIcon[regular]{globe}} 
        {\href{http://www.ubiq.ai/}{\normalsize \textbf{\underline{\textcolor{mainblue}{Ubiq}}}}}
        {}
        {\begin{itemize}
        \item Elaboration of spatial and temporal dynamic models for shared mobility demand-prediction. 
        \item Experiences on building predictive models for car and moped fleets in cities like Berlin, Budapest, Viena, Dubai, Washington DC, between others.  
        \item Large database pre-processing, engineering and preparation to the demand-prediction pipeline.
        \item Historic data analysis for reports and presentations with clients.
        \item Large datasets handling, management and information collection. 
        \item Participation in hiring processes. 
        \item \textbf{Main tech tools used}: R, SQL, FME, QGIS and Git.
        \end{itemize}}
        \\
	
    \twentyitem
   		{Sep 2015 -}
		{Dec 2016}
        {\textcolor{red}{Graduate Teaching Assistant}}
        {\href{http://www.uoguelph.ca}{University of Guelph}}
        {}
        {
        {\begin{itemize}
        \item \textcolor{red}{Item 1}
        \item \textcolor{red}{Item 2}
    \end{itemize}}
        }
     \\
     \twentyitem
   		{Dec 2013 -}
		{Apr 2015}
        {\textcolor{red}{Test Automation Engineer}}
        {\href{http://www.synechron.com/}{Synechron}}
        {}
        {
        \begin{itemize}
        \item \textcolor{red}{Item 1}
        \item \textcolor{red}{Item 2}
    \end{itemize}
    	}
        
	%\twentyitem{<dates>}{<title>}{<location>}{<description>}
\end{twenty}

%----------------------------------------------------------------------------------------
%	 RESEARCH
%----------------------------------------------------------------------------------------
\section{Research}
\begin{twenty}
	\twentyitem
    	{2016 - 2021}
		{}
        {PhD student}
        {\href{https://geographie.univie.ac.at/arbeitsgruppen/engage-geomorphologische-systeme-und-risikoforschung/}{ENGAGE group - UNIVIE}}
        {}
        {
       	\textbf{Thesis}: Landslide susceptibility mapping at national scale: first attempts for Austria. Scientific challenges within applicable solutions.
        {\begin{itemize}
        \item \textcolor{red}{Proposed a stepwise deterministic method to integrate datasets without labeled data. The method performs comparably with a method that incorporates a Support Vector Machine}
        \item \textcolor{red}{Prepared a longitudinal dataset to enable comprehensive analyses about WWI Canadian society and military, seeding further research}
        \item \textbf{Tech tools}: R, ArcGIS, QGIS, \large \LaTeX
		\end{itemize}}
        }
\end{twenty}

\section{Other interests}
\begin{center} Summer $\textbullet$ Outdoor activities $\textbullet$ Snowboard $\textbullet$ Maps $\textbullet$ Houseplants  \\ 
\vspace{2mm}
%http://mirror.neu.edu.cn/CTAN/fonts/fontawesome/doc/fontawesome.pdf
\faCoffee \hspace*{0.5cm} \faBeer \hspace*{0.5cm}\faSpotify \hspace*{0.5cm}\faSunO \hspace*{0.5cm}\faBicycle \hspace*{0.5cm}\faGraduationCap  \hspace*{0.5cm} \faMapMarker \hspace*{0.5cm}\faMap \hspace*{0.5cm}\faPlane \hspace*{0.5cm}\faRoad \hspace*{0.5cm}\faSignal \hspace*{0.5cm} \faLineChart  \hspace*{0.5cm}\faUniversity 
\end{center}

 %\vspace{2mm}

\end{document} 
